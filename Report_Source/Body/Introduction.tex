\section{Introduction}
Message Passing Interface (MPI) is a standardized and portable message-passing system. The standard only defines the expected syntax and semantics and not the implementation itself. MPI is used to facilitate the communication of distributed computational devices, each with their own memory. Thus, there is no shared global memory and all communication has to be done explicitly via messags. Thus, MPI is a communications protocol for programming parralele computers. In this paper, a test envrionment is emulated on a local machine to explore the uses of MPI. Experiments are run to explore and undercover differences between the the use of shared memory (As done via Threading) and the use of MPI as well as measure the differences in performance between MPI and threading solutions.

For this report. MPI was set up in a poin-to-point configuration. That is, there is no inter-slave communication and all communication is done via the master.
%\todo{Brief Background to MPI, It's use. The simulation envrionment and limitations}